\documentclass[uplatex, onecolumn, 10pt]{jsarticle}

\usepackage[dvipdfmx]{graphicx}
\usepackage{latexsym}
\usepackage{bmpsize}
\usepackage{url}
\usepackage{comment}
\usepackage{ulem}
\usepackage{amssymb}

\def\Underline{\setbox0\hbox\bgroup\let\\\endUnderline}
\def\endUnderline{\vphantom{y}\egroup\smash{\underline{\box0}}\\}

\newcommand{\ttt}[1]{\texttt{#1}}

\begin{document}

\title{\vspace{-40mm}\bf{\LARGE{ゼミ報告書}}}
\author{\vspace{-40mm}宅間 千隼  T2501200}
\date{2025-06-13 Fri}
\maketitle


\section{今日までにやったこと}

\subsection*{研究関連}
\begin{itemize}
	\item スカイコムさんMTG
	\begin{itemize}
        \item 6/12(木)
        \item 領域分割は今のままで,ラベル生成に集中することになりそう
        \item 具体的にどの入力(テキスト,図表)から,どういうラベルが生成したいのかは明確ではない
    \end{itemize}
	\item Pythonでの機械学習の勉強
\end{itemize}

\subsection*{その他}
\begin{itemize}
	\item ロボコン関連
	\begin{itemize}
	    \item 全体MTG
	    \item タスク
	    \item 技術フォローアップ
    \end{itemize}
	\item 医学部剣持研バイト
\end{itemize}



\section{次までにやること}

\subsection*{研究関連}
\begin{itemize}
    \item 研究実施計画書作成
    \item 関連研究調査
	\item Pythonでの機械学習の勉強
\end{itemize}

\subsection*{その他}
\begin{itemize}
	\item ロボコン関連
	\begin{itemize}
	    \item 全体MTG
	    \item タスク
	    \item モデルミーティング
    \end{itemize}
	\item 医学部剣持研バイト
\end{itemize}

\end{document}
