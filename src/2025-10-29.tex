\documentclass[uplatex, onecolumn, 10pt]{jsarticle}

\usepackage[dvipdfmx]{graphicx}
\usepackage{latexsym}
\usepackage{bmpsize}
\usepackage{url}
\usepackage{comment}

\def\Underline{\setbox0\hbox\bgroup\let\\\endUnderline}
\def\endUnderline{\vphantom{y}\egroup\smash{\underline{\box0}}\\}

\newcommand{\ttt}[1]{\texttt{#1}}

\begin{document}

\title{\vspace{-40mm}\bf{\LARGE{ゼミ報告書}}}
\author{\vspace{-40mm}宅間千隼 T2501200}
\date{2025-10-29 Wed}
\maketitle

\section{今日までにやったこと}

\subsection*{研究関連} 
\begin{itemize}
	\item データセット調査
	\begin{itemize}
		\item 柿木さんに聞き取りをした
		\begin{itemize}
		\item 修論前(10月あたり)からは自分のPCで作業していて、現在そのPCは人に譲ったため初期化済み
		\item Driveを確認していただいたところ、かなりデータを残せていなかった模様。
	\end{itemize}
	\item 修論時のデータセットがどれかもわからず、柿木さんのDriveにあるデータセットには、訓練、検証、テストデータ間で重複があり、計1629枚しかない。(修論には2161枚と書いてある。)
	\item 柿木さんの修論時の数値が正しいかも保障できないので、新しくモデルを作ることにした
	\item 作成する必要があるモデルのデータセットのうち、約850枚について、新規アノテーション、および、見直しをした
	\item モデル作成中(3/4個とりあえず作った)
	\end{itemize}
	\item 自分の拡張の方針変更
		\begin{itemize}
		\item 領域間の関係性抽出に生かせる拡張として、テキストの抽出精度向上(および正しいラベルを生成)と図表キャプション生成をすることにした
		\item 確実に成果が出て、評価もしやすいという点も考慮
		\item Gemini 2.0 Flash-Lite のAPI を用いて、テキスト抽出とキャプション生成を行う
		\item テキスト抽出の精度は関係性構築の上では欠かせないと判断し、精度が良い方法を優先
	    \end{itemize}
\end{itemize}

\subsection*{その他}
\begin{itemize}
	\item 講義・TA
	\item バイト
	\item ロボコン関連(ミーティング以外ほぼ手付かず)
\end{itemize}


\section{次までにやること}
\subsection*{研究関連}
\begin{itemize}
	\item スカイコムさんMTG(10/30(木))
	\item ICAROB
	\begin{itemize}
		\item 残りのモデル作成
		\item 評価
		\item 論文全体の流れとアブスト第一稿を10/31(金)までに片山先生に見ていただくことを目標。
	\end{itemize}
\end{itemize}

\subsection*{その他}
\begin{itemize}
    \item 講義・TA
	\item バイト
	\item ロボコン関連
\end{itemize}
\end{document}
