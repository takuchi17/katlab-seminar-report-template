\documentclass[uplatex, onecolumn, 10pt]{jsarticle}

\usepackage[dvipdfmx]{graphicx}
\usepackage{latexsym}
\usepackage{bmpsize}
\usepackage{url}
\usepackage{comment}

\def\Underline{\setbox0\hbox\bgroup\let\\\endUnderline}
\def\endUnderline{\vphantom{y}\egroup\smash{\underline{\box0}}\\}

\newcommand{\ttt}[1]{\texttt{#1}}

\begin{document}

\title{\vspace{-40mm}\textbf{\LARGE{ゼミ報告書}}}
\author{\vspace{-40mm}宅間千隼 T2501200}
\date{2025-12-17 Wed}
\maketitle

\section{今日までにやったこと}

\subsection*{研究関連} 
\begin{itemize}
    \item 11クラスでのデータセット構築
    \begin{itemize}
        \item text, title, author, table, figure, caption, pageHeader, pageFooter, sectionHeader, pageNumber, annotationの11クラス
        \item 619/1489完了(科学論文形式、帳票形式)
    \end{itemize}
    \item スカイコムさんMTG(12/11(木) 10:00~11:00)
    \begin{itemize}
        \item 今後の方針確認(11クラスデータセット作成、カタログっぽいデータセット探し)
        \item 今年度中に科学論文形式でグループ化できればうれしい。
    \end{itemize}
    \item 新しくできた分だけで学習を回してみた
    \begin{itemize}
        \item 少なくとも、どれも現在のモデルより精度が高い。1段階検出なので、速度も速い。
        \item モデルを変えたのに加えて、データセットのアノテーションをすべて見直した成果もある?
        \item RTMDet-tiny(YOLOシリーズを超えることを目的にOpenmmlabが開発した(2022)。Apache-2.0ライセンス)
        \begin{itemize}
            \item Openmmlabの事前学習物体検出モデルをファインチューニング
            \item "coco/bbox\_mAP": 0.796
            \item "coco/bbox\_mAP\_50": 0.985
            \item "coco/bbox\_mAP\_75": 0.944
        \end{itemize}
        \item RTMDet-small
        \begin{itemize}
            \item Openmmlabの事前学習物体検出モデルをファインチューニング
            \item "coco/bbox\_mAP": 0.82
            \item "coco/bbox\_mAP\_50": 0.992
            \item "coco/bbox\_mAP\_75": 0.959
        \end{itemize}
        \item RTMDet-small
        \begin{itemize}
            \item スクラッチ学習(ライセンス問題回避のため)
            \item "coco/bbox\_mAP": 0.762
            \item "coco/bbox\_mAP\_50": 0.979
            \item "coco/bbox\_mAP\_75": 0.907
        \end{itemize}
    \end{itemize}
\end{itemize}

\subsection*{その他}
\begin{itemize}
    \item 講義・TA
    \item バイト
    \item 研究室紹介
\end{itemize}


\section{次までにやること}
\subsection*{研究関連}
\begin{itemize}
    \item スカイコムさんMTG
    \begin{itemize}
        \item 12/25(木)10:00~11:00
    \end{itemize}
    \item データセット構築
    \item テキスト系領域の順序関係推論モデルのアーキテクチャ調査
\end{itemize}

\subsection*{その他}
\begin{itemize}
    \item 講義・TA
    \item バイト
    \item 研究室忘年会(12/19(金)17:30\textasciitilde 19:30)
    \item 12/21(日)-24(水)不在
\end{itemize}
\end{document}