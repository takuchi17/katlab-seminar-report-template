\documentclass[uplatex, onecolumn, 10pt]{jsarticle}

\usepackage[dvipdfmx]{graphicx}
\usepackage{latexsym}
\usepackage{bmpsize}
\usepackage{url}
\usepackage{comment}

\def\Underline{\setbox0\hbox\bgroup\let\\\endUnderline}
\def\endUnderline{\vphantom{y}\egroup\smash{\underline{\box0}}\\}

\newcommand{\ttt}[1]{\texttt{#1}}

\begin{document}

\title{\vspace{-40mm}\bf{\LARGE{ゼミ報告書}}}
\author{\vspace{-40mm}宅間千隼 T2501200}
\date{2025-10-08 Wed}
\maketitle

\section{今日までにやったこと}

\subsection*{研究関連} 
\begin{itemize}
	\item グルーピングの試行錯誤
	\begin{itemize}
		\item テキスト領域のみの場合ルールベースを使えそうだが、画像や表の割り当てがうまくいかない
		\item 画像や表の割り当てがキャプション生成器の精度に依存するため、単にテキストの内容でクラスタリングしても安定しない
		\item 参考にできそうな論文を見つけたので、読んでいるところ
		\begin{itemize}
			\item "GRAPH-BASED DOCUMENT STRUCTURE ANALYSIS"(Published as a conference paper at ICLR 2025)
			\item DocLayNetのデータセットを拡張して、領域ごとに他の領域との関係性をラベル付けしたデータセットを作成
				\item 上下左右、順序、親、子、参照
				\item 視覚情報のみ(テキストの内容は考慮していない)
				\item コードとデータセットは公開されていない?(GitHubにはREADMEしかない)
		\end{itemize}
	\end{itemize}
	\item AI領域分割後のルールベースの改良
	\begin{itemize}
		\item テキスト領域の背景が白以外だとルールベースがうまくいかない問題、新規作成する領域が何か1つにしか指定できない問題への対処
		\item テキスト領域の補正にOCR座標を、画像領域の補正には連結成分を用いる
		\item "text"と"figure"の2種類で生成できるようになった
		\item ただし、OCRを実行する必要があるため、計算時間はかなり増加する
	\end{itemize}
\end{itemize}

\subsection*{その他}
\begin{itemize}
	\item 玉置浩二ライブ(10/6(月)熊本城ホール)
	\begin{itemize}
	\item 座席は「1階1列目!!!」中央ブロック左寄り(手を伸ばせば届く距離)
	\item バンドなアレンジ
	\item セトリは、前半はコアなファンが喜ぶ曲、後半は有名な曲
	\item 三部構成で2時間
    \begin{enumerate}
		\item 本編
		\item トーク&弾き語り
		\item 公演終了アナウンス後にギター持って再登場\&合唱
	\end{enumerate}
	\end{itemize}
	\item 講義・TA
	\item バイト
\end{itemize}


\section{次までにやること}
\subsection*{研究関連}
\begin{itemize}
	\item ASLA
	\begin{itemize}
		\item スカイコムさんMTG(10/9(木) 10:00~11:00)
		\item 関連論文を参考に機械学習でのグルーピングを試す
		\item ICAROB計画(ルールベースの改良くらいしか成果がでていない)
	\end{itemize}
\end{itemize}

\subsection*{その他}
\begin{itemize}
    \item 10/14(火) 不在
	\item 玉置浩二新曲!
	\begin{itemize}
		\item 10/12(日)から始まる日曜劇場『ザ・ロイヤルファミリー』の主題歌
		\item 曲名「ファンファーレ」 
		\item <玉置浩二 コメント>

1年の半分以上が雪である、北海道で育った
自分自身の幼少期を重ねて作りました。
馬が小さい頃から育てられ、大きくなる姿は、自分の人生と重なります。
勝ち負けではなく、そのままの自分で真っ直ぐに前に向かって行くことが、大切なんだと思います。
ドラマ『ザ・ロイヤルファミリー』に関わる
すべてのみなさんの力になればと願っております。
この曲を聴いてくださる
すべてのみなさんの人生に祝福(ファンファーレ)あれ!
	\end{itemize}
\end{itemize}
\end{document}
