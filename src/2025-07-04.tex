\documentclass[uplatex, onecolumn, 10pt]{jsarticle}

\usepackage[dvipdfmx]{graphicx}
\usepackage{latexsym}
\usepackage{bmpsize}
\usepackage{url}
\usepackage{comment}

\def\Underline{\setbox0\hbox\bgroup\let\\\endUnderline}
\def\endUnderline{\vphantom{y}\egroup\smash{\underline{\box0}}\\}

\newcommand{\ttt}[1]{\texttt{#1}}

\begin{document}

\title{\vspace{-40mm}\bf{\LARGE{ゼミ報告書}}}
\author{\vspace{-40mm}宅間千隼 T2501200}
\date{2025-07-04 Fri}
\maketitle


\section{今日までにやったこと}

\subsection*{研究関連} 
\begin{itemize}
	\item 機械学習環境構築
	\begin{itemize}
	\item 一応Docker(CPU用、GPU用)で構築
	\item CPUとGPUのどっちを使うかを分岐で切り替えるようにした
	\item PyTorch、HuggingFaceのtransformers、MeCab、SudachiPy
	\end{itemize}
	\item 東北大の事前学習モデル(cl-tohoku/bert-base-japanese-v2)を使って、テキスト分類によるラベル生成をしてみた
	\begin{itemize}
	\item トークナイザーにはMeCabを使用(モデルに合わせて)
	\item 適当に遊んでみて、イメージはつかめた
	\end{itemize}
	\item 九州支部
	\begin{itemize}
	\item 申し込み
	\item 評価の章から書いてみている
	\end{itemize}
\end{itemize}

\subsection*{その他}
\begin{itemize}
	\item ロボコン関連
	\begin{itemize}
		\item ロボコンMTG
		\item タスク
		\item モデルMTG
	\end{itemize}
	\item 医学部剣持研究室バイト
\end{itemize}


\section{次までにやること}

\subsection*{研究関連}
\begin{itemize}
	\iten 研究室でかPCの環境構築
	\item Sudachiを前処理で使ってみる
	\item 研究計画書
	\begin{itemize}
	\item 関連研究
	\item どういうラベルを生成するのか
	\item ラベル生成の方法
	\item ASLAとどう組み合わせるか
	\end{itemize}
	\item スカイコムさんMTG
	\begin{itemize}
	\item 7/8(火) 11:00~12:00
	\end{itemize}
	\item 九州支部
	\begin{itemize}
	\item 評価・実装を書き進める
	\end{itemize}
\end{itemize}

\subsection*{その他}
\begin{itemize}
    \item ロボコン関連
		\begin{itemize}
			\item ロボコンMTG
			\item モデルMTG
			\item 試走会1に向けた調整
		\end{itemize}
    \item エンジニアリングコミュニケーションのポスター発表
    \begin{itemize}
		\item 7/7(月)16:40〜18:10
	\end{itemize}
\end{itemize}

\end{document}
