\documentclass[uplatex, onecolumn, 10pt]{jsarticle}

\usepackage[dvipdfmx]{graphicx}
\usepackage{latexsym}
\usepackage{bmpsize}
\usepackage{url}
\usepackage{comment}

\def\Underline{\setbox0\hbox\bgroup\let\\\endUnderline}
\def\endUnderline{\vphantom{y}\egroup\smash{\underline{\box0}}\\}

\newcommand{\ttt}[1]{\texttt{#1}}

\begin{document}

\title{\vspace{-40mm}\bf{\LARGE{ゼミ報告書}}}
\author{\vspace{-40mm}宅間千隼 T2501200}
\date{2025-08-04 Mon}
\maketitle


\section{今日までにやったこと}

\subsection*{研究関連} 
\begin{itemize}
	\item 画像キャプション生成
	\item HuggingFaceでの学習方法の調査
	\item スカイコムさんMTG(7/30(水) 13:00~14:00)
	\item 九州支部
	\begin{itemize}
		\item 飛行機、宿の予約
		\item 第一稿提出
		\item 第一稿修正
	\end{itemize}
\end{itemize}

\subsection*{その他}
\begin{itemize}
	\item ロボコン関連
	\begin{itemize}
		\item ロボコンMTG
		\item タスク
		\item モデルMTG
	\end{itemize}
	\item 医学部剣持研究室バイト
\end{itemize}


\section{次までにやること}

\subsection*{研究関連}
\begin{itemize}
	\item スカイコムさんMTG(8/13(水) 10:00~11:00)
	\item 学習データ作成用プログラムの作成
	\item 学習を試す
	\item 九州支部執筆
\end{itemize}

\subsection*{その他}
\begin{itemize}
    \item ロボコン関連
		\begin{itemize}
			\item ロボコンMTG
			\item モデルMTG
		\end{itemize}
	\item 医学部剣持研究室バイト
	\item 社会ニーズ
\end{itemize}

\end{document}
