\documentclass[uplatex, onecolumn, 10pt]{jsarticle}

\usepackage[dvipdfmx]{graphicx}
\usepackage{latexsym}
\usepackage{bmpsize}
\usepackage{url}
\usepackage{comment}

\def\Underline{\setbox0\hbox\bgroup\let\\\endUnderline}
\def\endUnderline{\vphantom{y}\egroup\smash{\underline{\box0}}\\}

\newcommand{\ttt}[1]{\texttt{#1}}

\begin{document}

\title{\vspace{-40mm}\bf{\LARGE{ゼミ報告書}}}
\author{\vspace{-40mm}宅間千隼 T2501200}
\date{2025-08-25 Mon}
\maketitle


\section{今日までにやったこと}

\subsection*{研究関連} 
\begin{itemize}
	\item スカイコムさんMTG(8/13(水) 10:00~11:00)
	\begin{itemize}
		\item 科学技術論文を対象にテキスト領域への意味ラベル付けを行う
		\item 章や節が意味ラベルとして有効なのでそれを割り当てるため、今回はAIというよりルールベースで実装する
		\item ページをまたいだ場合の領域結合
		\item 2段組みの場合の領域結合
	\end{itemize}
	\item 九州支部論文提出
\end{itemize}

\subsection*{その他}
\begin{itemize}
	\item ロボコン関連
	\begin{itemize}
		\item ロボコンMTG
		\item タスク
		\item モデルMTG
	\end{itemize}
	\item 医学部剣持研究室バイト
	\item 社会ニーズ
	\item ロボコン関連
	\item バイト
\end{itemize}


\section{次までにやること}

\subsection*{研究関連}
\begin{itemize}
	\item スカイコムさんMTG(8/26(火) 10:00~11:00)
	\item 1ページ内の同一意味領域結合と出力形式の検討
	\item 九州支部執筆プレゼン資料作成
\end{itemize}

\subsection*{その他}
\begin{itemize}
    \item ロボコン関連
		\begin{itemize}
			\item ロボコンMTG
			\item モデルMTG
		\end{itemize}
	\item バイト
\end{itemize}

\end{document}
