\documentclass[uplatex, onecolumn, 10pt]{jsarticle}

\usepackage[dvipdfmx]{graphicx}
\usepackage{latexsym}
\usepackage{bmpsize}
\usepackage{url}
\usepackage{comment}

\def\Underline{\setbox0\hbox\bgroup\let\\\endUnderline}
\def\endUnderline{\vphantom{y}\egroup\smash{\underline{\box0}}\\}

\newcommand{\ttt}[1]{\texttt{#1}}

\begin{document}

\title{\vspace{-40mm}\bf{\LARGE{ゼミ報告書}}}
\author{\vspace{-40mm}宅間千隼 T2501200}
\date{2025-09-03 Wed}
\maketitle


\section{今日までにやったこと}

\subsection*{研究関連} 
\begin{itemize}
	\item ASLA
	\begin{itemize}
	\item スカイコムさんMTG(8/26(火) 10:00~11:00)
	\begin{itemize}
		\item 図・表・テキスト領域の意味的グルーピングに決定
		\item 電子文書の種類は問わない
		\item MTGではマルチモーダルモデルを用いたクラスタリングを検討
	\end{itemize}
	\item グルーピング方法についての調査
	\begin{itemize}
		\item マルチモーダルモデルの調査
		\begin{itemize}
		\item 画像とテキストの両方から特徴埋め込みを生成するモデルに、CLIPやLayoutLMv3などがある
		\item まだ試せてはいないが、学習データをどう用意するかが課題
		\end{itemize}
		\item 段階的な手法でのグルーピング
		\begin{itemize}
		\item 図・表領域を画像として扱い、キャプション生成モデルおよびOCRで画像内の文字を取得してテキスト化
		\item テキスト領域と合わせてテキストベースのクラスタリングを行う
		\item 文章の意味的類似度を計算するモデルとして、Sentence-BERTを用いた
		\item ある程度はうまくいくが、OCRでの高精度なテキスト抽出が課題
		\end{itemize}
	\end{itemize}
	\end{itemize}
	\item 九州支部プレゼンテーション資料作成
\end{itemize}

\subsection*{その他}
\begin{itemize}
	\item ロボコン関連
	\begin{itemize}
		\item ロボコンMTG
		\item モデル提出
	\end{itemize}
	\item 社会ニーズ
	\item ロボコン関連
	\item バイト
\end{itemize}


\section{次までにやること}

\subsection*{研究関連}
\begin{itemize}
	\item ASLA
	\begin{itemize}
	\item マルチモーダルでのグルーピング手法の調査
	\item スカイコムさんMTG(9/4(火) 10:00~11:00)
	\end{itemize}
\end{itemize}

\subsection*{その他}
\begin{itemize}
    \item ロボコン関連
	\item 私的な予定で来週いっぱい忙しい
\end{itemize}

\end{document}
