\documentclass[uplatex, onecolumn, 10pt]{jsarticle}

\usepackage[dvipdfmx]{graphicx}
\usepackage{latexsym}
\usepackage{bmpsize}
\usepackage{url}
\usepackage{comment}

\def\Underline{\setbox0\hbox\bgroup\let\\\endUnderline}
\def\endUnderline{\vphantom{y}\egroup\smash{\underline{\box0}}\\}

\newcommand{\ttt}[1]{\texttt{#1}}

\begin{document}

\title{\vspace{-40mm}\bf{\LARGE{ゼミ報告書}}}
\author{\vspace{-40mm}宅間千隼 T2501200}
\date{2025-07-18 Fri}
\maketitle


\section{今日までにやったこと}

\subsection*{研究関連} 
\begin{itemize}
	\item スカイコムさんMTG
	\begin{itemize}
		\item 7/17(木) 15:00~16:00
		\item spaCyを実際に使っての報告
		\item 具体的な生成例はまだ提示されていない。企業(取引先?)に問い合わせているところ
		\item ブロックごとの意味ラベル生成も行う方針(ASLAの分割粒度)
		\item 次回ミーティングは7/30(水) 13:00~14:00に取り合えず決定
	\end{itemize}
	\item 九州支部
	\begin{itemize}
		\item 適用例を書いている途中
		\item 文章少しとできれば図を1つ載せたい
	\end{itemize}
\end{itemize}

\subsection*{その他}
\begin{itemize}
	\item ロボコン関連
	\begin{itemize}
		\item ロボコンMTG
		\item タスク
		\item 試走会1
		\item モデルMTG
	\end{itemize}
	\item 医学部剣持研究室バイト
\end{itemize}


\section{次までにやること}

\subsection*{研究関連}
\begin{itemize}
	\item ブロック単位での意味ラベル生成をBERTで試す(例えば、「introduction」「list」など)
	\item 意味ラベル生成関連研究調査
	\begin{itemize}
		\item 特に評価方法を知りたい
	\end{itemize}
	\item 九州支部
	\begin{itemize}
	\item 「適用例」の続きと「おわりに」の執筆
	\end{itemize}
\end{itemize}

\subsection*{その他}
\begin{itemize}
    \item ロボコン関連
		\begin{itemize}
			\item ロボコンMTG
			\item モデルMTG
		\end{itemize}
	\item 医学部剣持研究室バイト
	\item 7/23(水)から7/25(金)の間に木花キャンパスで、非常勤用の健康診断を受ける(23日午前の予定)
\end{itemize}

\end{document}
