\documentclass[uplatex, onecolumn, 10pt]{jsarticle}

\usepackage[dvipdfmx]{graphicx}
\usepackage{latexsym}
\usepackage{bmpsize}
\usepackage{url}
\usepackage{comment}

\def\Underline{\setbox0\hbox\bgroup\let\\\endUnderline}
\def\endUnderline{\vphantom{y}\egroup\smash{\underline{\box0}}\\}

\newcommand{\ttt}[1]{\texttt{#1}}

\begin{document}

\title{\vspace{-40mm}\bf{\LARGE{ゼミ報告書}}}
\author{\vspace{-40mm}宅間千隼 T2501200}
\date{2025-09-25 Thu}
\maketitle


\section{今日までにやったこと}

\subsection*{研究関連} 
\begin{itemize}
	\item ASLA
	\begin{itemize}
	\item スカイコムさんMTG(09/04(木) 10:00~11:00)
	\begin{itemize}
		\item カタログデータを頂いた
	\end{itemize}
	\item スカイコムさんMTG(09/25(木) 10:00~11:00)
	\begin{itemize}
		\item ASLAの領域分割での問題点
		\item グルーピングの方針
		\begin{itemize}
		  \item 構造的な文書は見出しを起点としたルールベースで対応
		  \item 非構造的な文書は、手動で領域分割したデータを用いて、いろいろ試す
	    \end{itemize}
	\end{itemize}
	\end{itemize}
	\item 九州支部(9/18(木))
\end{itemize}

\subsection*{その他}
\begin{itemize}
	\item ロボコン関連
	\begin{itemize}
		\item 調整
		\item 地区大会
	\end{itemize}
	\item 東京でインターン(9/4~9/13)
	\begin{itemize}
		\item NAIST生と話せて楽しかった
		\item 宮大OBの方(平古場さんの友達)と話せた
		\item Web系に踏ん切りがついた
		\item タダで行けた上にお金もらえた
	\end{itemize}
\end{itemize}


\section{次までにやること}

\subsection*{研究関連}
\begin{itemize}
	\item ASLA
	\begin{itemize}
		\item ソースコード共有
		\item データセット探し
		\item グルーピングの試行錯誤
	\end{itemize}
\end{itemize}

\subsection*{その他}
\begin{itemize}
    \item 9/27(土)~9/30(火) 不在
	\item この後最後の(?)ロボコンMTG
\end{itemize}

\end{document}
