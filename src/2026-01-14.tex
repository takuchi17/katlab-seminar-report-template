\documentclass[uplatex, onecolumn, 10pt]{jsarticle}

\usepackage[dvipdfmx]{graphicx}
\usepackage{latexsym}
\usepackage{bmpsize}
\usepackage{url}
\usepackage{comment}

\def\Underline{\setbox0\hbox\bgroup\let\\\endUnderline}
\def\endUnderline{\vphantom{y}\egroup\smash{\underline{\box0}}\\}

\newcommand{\ttt}[1]{\texttt{#1}}

\begin{document}

\title{\vspace{-40mm}\textbf{\LARGE{ゼミ報告書}}}
\author{\vspace{-40mm}宅間千隼 T2501200}
\date{2026-01-14 Wed}
\maketitle

\section{今日までにやったこと}

\subsection*{研究関連} 
\begin{itemize}
    \item 11クラスでの物体検出データセット構築
    \begin{itemize}
        \item text, title, author, table, figure, caption, pageHeader, pageFooter, sectionHeader, pageNumber, annotationの11クラス
        \item 科学論文、スライド、帳票データセットは完了
    \end{itemize}
    \item DocLayNetデータセットで事前学習モデル(RTMDet)を作ってみている
    \item 図、表、注釈、キャプションのグルーピングの正解データセットの作成とルールベースでのグルーピングを実験中
\end{itemize}

\subsection*{その他}
\begin{itemize}
    \item 講義・TA
    \item バイト
\end{itemize}


\section{次までにやること}
\subsection*{研究関連}
\begin{itemize}
    \item スカイコムさんMTG
    \begin{itemize}
        \item 1/22(木)10:00~11:00 (仮)
    \end{itemize}
    \item 親子関係の正解データセット構築
    \item ルールベースでのグルーピング(要素関係の親子関係)
\end{itemize}

\subsection*{その他}
\begin{itemize}
    \item 講義・TA
    \item バイト
\end{itemize}
\end{document}