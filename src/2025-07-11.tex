\documentclass[uplatex, onecolumn, 10pt]{jsarticle}

\usepackage[dvipdfmx]{graphicx}
\usepackage{latexsym}
\usepackage{bmpsize}
\usepackage{url}
\usepackage{comment}

\def\Underline{\setbox0\hbox\bgroup\let\\\endUnderline}
\def\endUnderline{\vphantom{y}\egroup\smash{\underline{\box0}}\\}

\newcommand{\ttt}[1]{\texttt{#1}}

\begin{document}

\title{\vspace{-40mm}\bf{\LARGE{ゼミ報告書}}}
\author{\vspace{-40mm}宅間千隼 T2501200}
\date{2025-07-11 Fri}
\maketitle


\section{今日までにやったこと}

\subsection*{研究関連} 
\begin{itemize}
	\item スカイコムさんMTG
	\item 7/8(火) 11:00~12:00
	\item どういうラベルを生成するかの検討
	\item spaCyという自然言語処理ライブラリを知った
\end{itemize}
\begin{itemize}
	\item spaCy + SudachiPyを使ってみた
	\begin{itemize}
	\item MeCabよりも分かち書きの粒度が柔軟(国際/連合/安全/保障/理事/会 → 国際連合/安全保障/理事会)
	\item 固有表現抽出と分類の精度が高い
	\begin{itemize}
		\item Person、Money、Date、Provinceなどを分類できた
		\item 文脈での意味は考慮されていない(日付は日付でも、「開催日」なのか「締め切り日」なのかなど)
		\item spaCyで分類して、その周辺の文脈を専用のモデルに入力すると、効率的に意味ラベルを生成できそう
	\end{itemize}
\end{itemize}
\item 東北大のBERTモデルのトークナイザ―にSudachiを使ってみた
\begin{itemize}
	\item MeCabを前提に学習されているため、やはり精度が低い(学習データが少ない場合のみ実験)
\end{itemize}
	\item 九州支部
	\begin{itemize}
	\item 考察、実装、はじめにを書いた
	\item スペース的におわりにと適用例が入らないかもしれないため、実装を工夫する。
	\end{itemize}
\end{itemize}

\subsection*{その他}
\begin{itemize}
	\item ロボコン関連
	\begin{itemize}
		\item ロボコンMTG
		\item タスク
		\item モデルMTG
	\end{itemize}
	\item 医学部剣持研究室バイト
\end{itemize}


\section{次までにやること}

\subsection*{研究関連}
\begin{itemize}
	\item 研究室でかPCの環境構築
	\item スカイコムさんMTG
	\begin{itemize}
	\item 7/15(火) 11:00~12:00
	\item 具体例の図を作成して臨む
	\end{itemize}
	\item 九州支部
	\begin{itemize}
	\item 入らなくても、適用例、おわりに、参考文献をとりあえず書く
	\end{itemize}
\end{itemize}

\subsection*{その他}
\begin{itemize}
    \item ロボコン関連
		\begin{itemize}
			\item ロボコンMTG
			\item モデルMTG
			\item 試走会1
		\end{itemize}
\end{itemize}

\end{document}
