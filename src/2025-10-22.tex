\documentclass[uplatex, onecolumn, 10pt]{jsarticle}

\usepackage[dvipdfmx]{graphicx}
\usepackage{latexsym}
\usepackage{bmpsize}
\usepackage{url}
\usepackage{comment}

\def\Underline{\setbox0\hbox\bgroup\let\\\endUnderline}
\def\endUnderline{\vphantom{y}\egroup\smash{\underline{\box0}}\\}

\newcommand{\ttt}[1]{\texttt{#1}}

\begin{document}

\title{\vspace{-40mm}\bf{\LARGE{ゼミ報告書}}}
\author{\vspace{-40mm}宅間千隼 T2501200}
\date{2025-10-22 Wed}
\maketitle

\section{今日までにやったこと}

\subsection*{研究関連} 
\begin{itemize}
	\item スカイコムさんMTG(10/15(木) 11:00~12:00)
	\begin{itemize}
		\item LayoutLMv3で関係性を学習をやってみる方針で進める
		\item まずは同一ページ内で
	\end{itemize}
	\item ICAROB
	\begin{itemize}
		\item 方針を決めるために柿木さんのこれまでの論文を確認
	    \begin{itemize}
			\item ICAROB: 190枚の日本語の科学技術論文データセットで学習したモデルを使用。ルールベースの提案。maskrcnnからcascadercnnへ変更。
			\item ICEGC: ASLAという言葉は出していなく、文書分類→各文書形式に特化したモデルで推論することを単独で提案。データセットもDocLayNetと、ASLAが用いてきた日本語文書ではない。レイアウト解析モデルは科学技術論文のみ。
			\item JRNAL: ICAROBと流れは同じ。データセットはDocLayNetなので、日本語文書とは言っていない。ただ、適用例は日本語文書の画像。
			\item 紀要: 既存のASLAにルールベースを提案、Maskrcnn→cascadercnnへ変更。日本語文書のデータセットを増やした。
			\item 修論: ICAROBから、ルールベースとcascadercnnへ変更した拡張を除いたものを既存のASLAとしている。拡張後はICGECで提案している文書形式モデル→4つの特化Cascadeモデルでのレイアウト解析、ルールベースを追加。
	    \end{itemize}
		\item 現状、今年のICAROBでの方針は次の3パターンを考えている。ここで注意したいのは、柿木さんが残しているASLAは修論時のみのもの。
		\begin{itemize}
			\item 既存のASLAのルールベースのみに着目し、背景色がある文書を含んで、新ルールベースとの評価をする。
			\item 柿木さんの修論の内容を踏襲し、既存のASLAをICAROBで更新しておく
			\item 柿木さんの修論の内容を踏襲し、既存のASLAをICAROBで更新する + 既存の2値化とルールベースの改良により適用範囲を拡大
		\end{itemize}
	\end{itemize}
	\item データセット探し、機能しないモデルの調査
	\begin{itemize}
		\item 論文を改めて読むと、柿木さんが残しているデータセットは、GUIアノテーションツールで作った元データではなく、学習時用に変換したもののみ。
		\item 既存のASLAとの比較に使っていたデータがどれかわからない状態。
		\item スカイコムさんに調査していただいた。
		\item 研究室共有の2台のPC内を漁った。→何とか既存のASLAのデータセットは検索を駆使して見つかった。が、学習用のアノテーションファイルが散らばっている。クラスも同じかわからない。
		\item 柿木さんがgraduateに残している機械学習モデルのうち一つが、推論しても何も検出しない。
	\end{itemize}
\end{itemize}

\subsection*{その他}
\begin{itemize}
	\item 講義・TA
	\item バイト
	\item ロボコン関連
\end{itemize}


\section{次までにやること}
\subsection*{研究関連}
\begin{itemize}
	\item ICAROBのためのテストデータ探し、ないものはアノテーションを付ける
	\item ICAROB方針固め
	\item スカイコムさんMTG(10/30(木))
\end{itemize}

\subsection*{その他}
\begin{itemize}
    \item 講義・TA
	\item バイト
	\item ロボコン関連
\end{itemize}
\end{document}
