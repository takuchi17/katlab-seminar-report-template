\documentclass[uplatex, onecolumn, 10pt]{jsarticle}

\usepackage[dvipdfmx]{graphicx}
\usepackage{latexsym}
\usepackage{bmpsize}
\usepackage{url}
\usepackage{comment}

\def\Underline{\setbox0\hbox\bgroup\let\\\endUnderline}
\def\endUnderline{\vphantom{y}\egroup\smash{\underline{\box0}}\\}

\newcommand{\ttt}[1]{\texttt{#1}}

\begin{document}

\title{\vspace{-40mm}\bf{\LARGE{ゼミ報告書}}}
\author{\vspace{-40mm}宅間千隼 T2501200}
\date{2025-05-23 Fri}
\maketitle


\section{今日までにやったこと}

\subsection*{研究関連} 
\begin{itemize}
	\item スカイコムさんMTG
	\item ASLA推論環境構築
	\begin{itemize}
		\item 柿木さん、萩山さんのREADMEでほとんどうまくいった
		\item 柿木さんのREADMEには、requirements.txtがあるとあったが、gitignoreされていて、かつおそらく含め忘れていたので、Pythonのエラー頼りにパッケージをインストールしていった
		\item Visual C++のランタイムが必要だった(SkyViewCOM)
	\end{itemize}
	\item ASLAを動かしてみた
	\begin{itemize}
		\item 自分の卒論
		\item ロボコン規約
		\item ラベルは、卒論の日本語がすべて文字化けしていた。ロボコン規約のラベルはよく取れた。
	\end{itemize}
\end{itemize}

\subsection*{その他}
\begin{itemize}
	\item ロボコン関連
	\begin{itemize}
		\item ロボコンMTG
		\item モデルMTG, モデル要求
		\item 実装・レビュー
	\end{itemize}
	\item 医学部剣持研究室バイト
	\begin{itemize}
		\item 月曜:17:00~19:00
		\item 火曜:15:00~17:00
	\end{itemize}
	\item eAPRIN受講\&修了証提出
	\item TA最終提出物提出
\end{itemize}


\section{次までにやること}

\subsection*{研究関連}
\begin{itemize}
	\item ASLA論文読み
	\item ASLA現状整理
	\item 英論発表準備
\end{itemize}

\subsection*{その他}
\begin{itemize}
    \item ロボコン関連
	\begin{itemize}
		\item ロボコンMTG
		\item モデルMTG, モデル要求(5月いっぱい)
		\item 2024年度他チームモデル読み
		\item 実装・レビュー
	\end{itemize}
    \item 医学部剣持先生のバイト
    \begin{itemize}
		\item 月曜:17:00~19:00
		\item 火曜:15:00~17:00
	\end{itemize}
\end{itemize}

\end{document}
