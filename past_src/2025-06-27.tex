\documentclass[uplatex, onecolumn, 10pt]{jsarticle}

\usepackage[dvipdfmx]{graphicx}
\usepackage{latexsym}
\usepackage{bmpsize}
\usepackage{url}
\usepackage{comment}

\def\Underline{\setbox0\hbox\bgroup\let\\\endUnderline}
\def\endUnderline{\vphantom{y}\egroup\smash{\underline{\box0}}\\}

\newcommand{\ttt}[1]{\texttt{#1}}

\begin{document}

\title{\vspace{-40mm}\bf{\LARGE{ゼミ報告書}}}
\author{\vspace{-40mm}宅間千隼 T2501200}
\date{2025-06-27 Fri}
\maketitle


\section{今日までにやったこと}

\subsection*{研究関連} 
\begin{itemize}
	\item 計画書草案の作成
	\item スカイコムさんMTG:6/24(火) 11:00~
	\item 九州支部タイトル・アブスト作成
\end{itemize}

\subsection*{その他}
\begin{itemize}
	\item ロボコン関連
	\begin{itemize}
		\item ロボコンMTG
		\item タスク
		\item モデルMTG
	\end{itemize}
	\item 医学部剣持研究室バイト
    \item エンジニアリングコミュニケーション3回+先生(完)
\end{itemize}


\section{次までにやること}

\subsection*{研究関連}
\begin{itemize}
	\item ラベル生成の関連研究調査
	\item 生成するラベルの具体例の作成
	\item 領域分割とのかけ合わせで何かできないか検討
	\item Sudachiを触る
	\item 九州支部タイトル・アブスト提出
	\item 九州支部執筆環境構築
\end{itemize}

\subsection*{その他}
\begin{itemize}
    \item ロボコン関連
		\begin{itemize}
			\item ロボコンMTG
			\item モデルMTG
			\item 試走会1に向けた調整
			\item 技術フォロー
		\end{itemize}
    \item 医学部剣持先生のバイト
\end{itemize}

\end{document}
